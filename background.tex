\section{Background}
Adblock plus is one of the widely use adblockers.
The Firefox add on alone has b403,191,426 downloads in total with 4,375,399 happening in the last 30 days ~\cite{abpmoz}.
Adblock plus relies  on filter lists for  its functioning.
The filter lists can be categorized into whitelists and blacklists.
The filters are nothing but text files that  have  a set of rules.
Blacklists contain rules for blocking advertisements whereas whitelists acts as exception for the rules that are in the blacklists.
The filters are  published and maintained by easylist~\cite{easylist}.
The rules in the white and blacklists are CSS and Xpath selectors that are prefixed with a predefined set of symbols.These rules interpreted as regular expressions by the adblock plus engine.
The notation is that filters prefixed with an $"@@"$ is a whitelisting filter and  filters prefixed with $"\#\#"$
or $\#@$  is  a blocking or blacklisting filter.
The blocking filters can be broadly classified into:
\begin{enumerate}
\item Element Hiding filters
\item URL filters
\end{enumerate}
Element hiding filters hide the elements present in an webpage from the user whereas the URL filters block the web page from requesting a resource , be it a webpage or javascript.
