\section{Background}
Adblock plus is one of the widely use adblockers.  <statistics>
Adblock plus relies  on filter lists for  its functioning.
The filter lists can be categorized into whitelists and blacklists.
Blacklists contain rules for blocking advertisements whereas whitelists acts as exception for the rules that are in the blacklists.
The rules are CSS and Xpath selectors that are prefixed with a predefined set of symbols.
These rules interpreted as regular expressions by the adblock plus engine.
The notation is that filters prefixed with an $"@@"$ is a whitelisting filter and  filters prefixed with $"\#\#"$
or $\#@$  is  a blocking or blacklisting filter.
The blocking filters can be broadly classified into:
\begin{enumerate}
\item Element Hiding filters
\item URL filters
\end{enumerate}
Element hiding filters hide the elements present in an webpage from the user whereas the URL filters block the web page from requesting a resource , be it a webpage or javascript.
The filters are  published and maintained by easylist~\cite{easylist}.