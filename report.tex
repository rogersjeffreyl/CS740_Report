% This is "sig-alternate.tex" V2.1 April 2013
% This file should be compiled with V2.5 of "sig-alternate.cls" May 2012
%
% This example file demonstrates the use of the 'sig-alternate.cls'
% V2.5 LaTeX2e document class file. It is for those submitting
% articles to ACM Conference Proceedings WHO DO NOT WISH TO
% STRICTLY ADHERE TO THE SIGS (PUBS-BOARD-ENDORSED) STYLE.
% The 'sig-alternate.cls' file will produce a similar-looking,
% albeit, 'tighter' paper resulting in, invariably, fewer pages.
%
% ----------------------------------------------------------------------------------------------------------------
% This .tex file (and associated .cls V2.5) produces:
%       1) The Permission Statement
%       2) The Conference (location) Info information
%       3) The Copyright Line with ACM data
%       4) NO page numbers
%
% as against the acm_proc_article-sp.cls file which
% DOES NOT produce 1) thru' 3) above.
%
% Using 'sig-alternate.cls' you have control, however, from within
% the source .tex file, over both the CopyrightYear
% (defaulted to 200X) and the ACM Copyright Data
% (defaulted to X-XXXXX-XX-X/XX/XX).
% e.g.
% \CopyrightYear{2007} will cause 2007 to appear in the copyright line.
% \crdata{0-12345-67-8/90/12} will cause 0-12345-67-8/90/12 to appear in the copyright line.
%
% ---------------------------------------------------------------------------------------------------------------
% This .tex source is an example which *does* use
% the .bib file (from which the .bbl file % is produced).
% REMEMBER HOWEVER: After having produced the .bbl file,
% and prior to final submission, you *NEED* to 'insert'
% your .bbl file into your source .tex file so as to provide
% ONE 'self-contained' source file.
%
% ================= IF YOU HAVE QUESTIONS =======================
% Questions regarding the SIGS styles, SIGS policies and
% procedures, Conferences etc. should be sent to
% Adrienne Griscti (griscti@acm.org)
%
% Technical questions _only_ to
% Gerald Murray (murray@hq.acm.org)
% ===============================================================
%
% For tracking purposes - this is V2.0 - May 2012

\documentclass{sig-alternate-05-2015}
\usepackage{amsmath,epsfig}
\usepackage{graphicx}
\usepackage{caption}
\usepackage{algorithm}
\usepackage[noend]{algpseudocode}
\usepackage[center]{subfigure}
\usepackage{color,graphicx}
\usepackage{multirow}
\usepackage{cite}
\usepackage{textcomp}
\usepackage{amssymb}
\usepackage{algorithm}
\usepackage[noend]{algpseudocode}
\usepackage{listings}
\makeatletter
\def\BState{\State\hskip-\ALG@thistlm}
\makeatother
\begin{document}

% Copyright

%\setcopyright{acmlicensed}
%\setcopyright{rightsretained}
%\setcopyright{usgov}
%\setcopyright{usgovmixed}
%\setcopyright{cagov}
%\setcopyright{cagovmixed}
%
% --- Author Metadata here ---

%\CopyrightYear{2007} % Allows default copyright year (20XX) to be over-ridden - IF NEED BE.
%\crdata{0-12345-67-8/90/01}  % Allows default copyright data (0-89791-88-6/97/05) to be over-ridden - IF NEED BE.
% --- End of Author Metadata ---

\title{Can we refine the notion of what an acceptable ad is?}
\subtitle{}
%
% You need the command \numberofauthors to handle the 'placement
% and alignment' of the authors beneath the title.
%
% For aesthetic reasons, we recommend 'three authors at a time'
% i.e. three 'name/affiliation blocks' be placed beneath the title.
%
% NOTE: You are NOT restricted in how many 'rows' of
% "name/affiliations" may appear. We just ask that you restrict
% the number of 'columns' to three.
%
% Because of the available 'opening page real-estate'
% we ask you to refrain from putting more than six authors
% (two rows with three columns) beneath the article title.
% More than six makes the first-page appear very cluttered indeed.
%
% Use the \alignauthor commands to handle the names
% and affiliations for an 'aesthetic maximum' of six authors.
% Add names, affiliations, addresses for
% the seventh etc. author(s) as the argument for the
% \additionalauthors command.
% These 'additional authors' will be output/set for you
% without further effort on your part as the last section in
% the body of your article BEFORE References or any Appendices.

\numberofauthors{2} %  in this sample file, there are a *total*
% of EIGHT authors. SIX appear on the 'first-page' (for formatting
% reasons) and the remaining two appear in the \additionalauthors section.
%
\author{
% You can go ahead and credit any number of authors here,
% e.g. one 'row of three' or two rows (consisting of one row of three
% and a second row of one, two or three).
%
% The command \alignauthor (no curly braces needed) should
% precede each author name, affiliation/snail-mail address and
% e-mail address. Additionally, tag each line of
% affiliation/address with \affaddr, and tag the
% e-mail address with \email.
%
% 1st. author
\alignauthor
Arjun Gurumurthy\\
       \affaddr{Dept Of Computer Sciences}\\
       \affaddr{UW-Madison}\\
       \affaddr{Madison,Wisconsin}\\
       \email{arjun@cs.wisc.edu}
% 2nd. author
\alignauthor
Rogers Jeffrey\\
       \affaddr{Dept Of Computer Sciences}\\
       \affaddr{UW-Madison}\\
       \affaddr{Madison,Wisconsin}\\
       \email{rl@cs.wisc.edu}
}
\maketitle
\begin{abstract}
Ad blockers have caused an increasing amount of debate over the recent years, with the scope of arguments ranging from their economic impact to ethical considerations. A majority of ad blockers rely on whitelists and blacklists of websites to control the ads being displayed.
Adblock Plus, one of the widely used ad blockers, has enumerated a set of rules that an ad displaying page should follow for an advertisement to be determined as acceptable or not. In this project we investigate the behavior of adblock plus to  .Our study begins with the analysis of the ad-block plus filtering-mechanisms. Existing addons provide us with capability to understand the filtering process. We then analyze the behavior of adblock plus on 1500 webpages to see how adblock plus implements or enforces the acceptable ads criteria. We use our analysis to see if minor modifications to the page structure can cause the ads to pass through the filtering process of the adblocker. Our results show that adblock plus is too restrictive as it blocks non advertising related content. We also observe that adblock plus can be obviated by minor modifications to the web pages.
 \end{abstract}

%\terms{Theory}

\keywords{Ad Fraud,Impression Fraud,Acceptable Ads}

\section{Introduction}
Content and services which are offered for free on the Internet are sometimes monetized through online advertisements.
This relies on the implicit understanding that the viewers who consume free content do so at the cost of viewing ads.
However, recent times have seen the  rise of ad blockers that control the display of  advertisements on websites. These adblockers rely on  a list of filters for blocking or allowing advertisements.
In 2011 adblock plus started monetization of the whitelist by launching what is known was an "Acceptable ads" program.As a part of the acceptable ads program adblock plus proposed  that ads  displayed on a page have to adhere to. A publisher adhering to this crtieria can pay a fee to adblock plus and get themselves whitelisted. It is known that the e-commerce and advertising giants like Amazon, Google and Microsoft paid an undisclosed sum to adblock plus to get themselves whitelisted.

While the number of users of adblock plus continues to grow, on the other hand the number of exception filters added to the whitelist has also grown over the years.
This raises the following questions.
\begin{itemize}
\item [Q1.] How does whitelisting impact the number of ads blocked? \label{q:q1}
\item [Q2.] Does Adblock Plus allow ads even if they do not satisfy the criteria of an acceptable ad? \label{q:q2}
\item [Q3.] Does adblock Plus block ads even if they satisfy the four criteria? \label{q:q3}
\item [Q4.] As more sites are added to the whitelist when can adblock plus be deemed in-effective? \label{q:q4}
\end{itemize}
We attempt to answer Q1,Q2 and Q3 in our project by performing crawling web pages from different sources and analyzing the behavior of adblock plus on these webpages.


\section{Background}
Adblock plus is one of the widely use adblockers.  <statistics>
Adblock plus relies  on filter lists for  its functioning.
The filter lists can be categorized into whitelists and blacklists.
Blacklists contain rules for blocking advertisements whereas whitelists acts as exception for the rules that are in the blacklists.
The rules are CSS and Xpath selectors that are prefixed with a predefined set of symbols.
These rules interpreted as regular expressions by the adblock plus engine.
The notation is that filters prefixed with an $"@@"$ is a whitelisting filter and  filters prefixed with $"\#\#"$
or $\#@$  is  a blocking or blacklisting filter.
The blocking filters can be broadly classified into:
\begin{enumerate}
\item Element Hiding filters
\item URL filters
\end{enumerate}
Element hiding filters hide the elements present in an webpage from the user whereas the URL filters block the web page from requesting a resource , be it a webpage or javascript.
The filters are  published and maintained by easylist~\cite{easylist}.
\section{{\secit Related Work}}
 General aspects of adblocking have been discussed in various studies. These studies have focussed on economic impact that adblockers have on advertising.  \cite{DigitalTrends} suggests that adblock railroads publishers and content providers to pay for their sites to be whitelisted.
\cite{pujol2015annoyed} describe a method of classifying ad traffic by leveraging adblock plus. They use libadblock plus to classify the web traffic based on hits from the easylist and the whitelist. They also note that it is not possible to associate HTTP traffic with ad objects and that information is needed about the structure of the web page to achieve higher accuracy in detection of ad traffic. Finally, they establish that it is not possible to identify hidden ads without knowledge of the html page content.
By aiming to look at whether advertisements are acceptable or not, we are not just dealing with fraudulent ads that are propagated across the Internet, but also ads that a user would perceive as annoying or intrusive. In our process of understaning what an acceptable ad is, we look into existing definitions of what it means to be an acceptable advertisement.
\cite{walls2015measuring} describe criteria for an acceptable ad as defined by Adblock plus:
\begin{itemize}
\item Advertisements cannot contain animations, sounds, or "attention grabbing" images.
\item Advertisements cannot obscure page content or obstruct reading flow, i\.e\., the ad cannot be placed in the middle of a block
of text.
\item Advertisements must be clearly distinguished from the page
content and must be labeled using the word "advertisement"
or equivalent terms.
\item Banner advertisements should not force the user to scroll
down to view page content.
\cite{walls2015measuring} further present a comprehensive study of adblock plus and  an analysis of how the users perceive acceptable ads. They report that $ 90 \% $ of the users viewing an all grid layout ads could not distinguish them from the content. Allowing this ads thus seems to be in conflict with adblocks acceptable ads policies.
\end{itemize}

Furthermore , in their analysis of PPV networks \cite{springborn2013impression} describe a viewport size filter mechanism (a viewport is the user's visible area of a web page) for detecting ad views that are too small to be seen by the user.

It can be clearly seen that the html content of the web page has a ton of information that can be used to characterize the nature of the ad being displayed in the web page.
We thus concentrate our discussion on analysis of web page structure to detect and redefine unacceptable ads.

\section{Methodology}
We investigate the working of AdBlock Plus, and see how the results it provide could give insights about ad blocking mechanisms in general. We try four different kinds of experiments to see how AdBlock Plus works:
\begin{enumerate}
	\item[(a)] 

The impact of whitelists on AdBlock Plus's blocking mechanisms 
\item [(b)]
 The change in whitelist over time
\item[(c)]
 If we do away with the notion of lists, can we see how AdBlock Plus performs with respect to element hiding filters alone.
 \item[(d)] Fuzzing the page to prevent AdBlock Plus from recognizing the ads in the page.

\end{enumerate}

 AdBlock Plus provides two means of configuring the whitelist and blacklist in its Options dialog -  one is to add the filters manually, the other is to provide a URL that has the list of filters. By default, the easylist.txt URL is added, and if the Acceptable Ads program is allowed, the exception\_rules.txt URL is also added.


Crawling and obtaining ads: The initial problem was how to get the ads from the websites, and how to get the ads that were blocked. There is a Firefox add-on called AdBlock Plus Element Hiding Helper that enables us to analyze the ads that were blocked - essentially - it marks the element as blocked, denotes the content type of the element, and provides the maching URL/hiding filter that caused it to be blocked. 

This allows the user to analyze and see the format of ads that were blocked, whitelisted or hidden using element hiding filters. With this in hand, in addition we needed a new mechanism to get the ads blocked and ads shown given a page. In addition we needed a way by which we could use different configurations for whitelist, blacklist and element hiding filters and get the ad details for all the configurations. We modified the abpcrawler tool to do the same. This enhanced abpcrawler working is outlined in the following algorithm:
1. Retrieve the list of configurations for blacklist and whitelist
2. For each configuration in 1, do the following
3. Launch firefox with the AdBlock Plus add-on set to the particular configuration.
4. In JSON format, write to a file the following:
a. The content type of the element
b. The filter regex applied (null if no filter is applied)
c. The location of the element within the page.
5. Return.

Thus for our experiments, the configuration in step 2 alone needs to be changed. For example, to see the impact of the change in whitelist (a), the corresponding whitelist URL is configured, and the crawler is run. 

Measuring performance of ads:
One key aspect to measure is to see the impact of URL filters on AdBlock Plus's performance. We expose a new URL that has only the element hiding filters (a subset of easy\_list.txt) and configure it to run the crawler. We can now test the performance of AdBlock Plus with three configurations - the ‘normal’ AdBlock Plus (the default typical configuration, with Acceptable Ads enabled), EasyList alone (Acceptable Ads disabled) and Element Hiding (only element hiding filters), and measure the number of ads blocked. 
A naive measurement of the number of ads blocked is to count the number of non-null and non-allowed filters in the JSON file. This doesn’t always work - with whitelist and blacklist, the number of ads blocked shows a higher number than with only the blacklist. The reason for this is that when URLs are blocked, the call never gets through and hence the number of elements in the page is considerably less. This would mean that even though more elements are explicitly blocked, the actual number of blocked elements is higher.
To get the actual measurement, we have a new measure, the total number of elements in the page.

The procedure is as follows
Get the total number of elements in the page
For each configuration,
Blocked elements in page = Get the count of non-null and non-allowed filters in the page.
Effectively blocked elements = Total number of elements (1) - Element count in page for current configuration
Total blocked elements = Blocked elements in page+Effectively blocked elements.

This is then repeated for multiple configurations.

In addition, to check whether it is possible to circumvent the ads, we do the following:
Circumventing element hiding filters:
AdBlock Plus blocks ads by looking into the blacklist and whitelist and hiding elements that are prefixed with the element hiding filter tag. This means that the blocking mechanism can be considered as a form of string matching process. Any way to circumvent the string matching process would then allow the ad to be displayed. One method we followed was to use a URL redirection filter - this would result in the element to not be blocked as the string would not be matched. 
Blocking of Non-Ad elements
One other experiment would be to check if elements that are not ads are blocked by AdBlock Plus







\section{Data}
\begin{table}[]
\label{tab:website}
\begin{tabular}{|l|l|}
\hline
Domain         & Count of pages crawled \\ \hline
digitalspy     & 40                     \\ \hline
theguardian    & 40                     \\ \hline
variety        & 39                     \\ \hline
indiewire      & 23                     \\ \hline
hitfix         & 20                     \\ \hline
cinemablend    & 20                     \\ \hline
darkhorizons   & 20                     \\ \hline
empireonline   & 20                     \\ \hline
eonline        & 20                     \\ \hline
etonline       & 20                     \\ \hline
fandango       & 20                     \\ \hline
tmz            & 20                     \\ \hline
flickeringmyth & 20                     \\ \hline
goldderby      & 20                     \\ \hline
heyuguys       & 20                     \\ \hline
\end{tabular}
\centering
\caption{Top-15 Crawled Websites}
\end{table}

For our initial experiments on  undertanding the behavior of adblock we crawled webpages from  200 websites.
These websites were picked from the  top 500 websites as ranked by Alexa ~\cite{alexa} and included sites from the following domains: sports, shopping, news.

For measuring the impact of the whitelist and blacklist over time, we used the Wayback Machine from the Internet Archive ~\cite{wayback} to obtain the revisions of whitelist and blacklist from 2012 to 2016.
There were 72 revisions in total for the whitelist and 69 revisions for the blacklist. We obtained the whitelist and blacklist at the end of each year from 2012 to 2016.

To find the number of blocked elements for each filter combination, we required websites that had a relatively large number of ads.
This was needed to obtain a clear picture of the performance of AdBlock Plus on such ad-intensive pages.
The Newsdesk page of IMDB is a hub that links to many such websites. We crawled 1500 webpages from 130 websites.
Table ~\ref{tab:website} lists the top-15 websites and the webpage counts.

\section{Results}
Figure~\ref{fig:abp-filters} indicates the different types of filters  present in adblock.
The horizontal axis lists the domains associated with the webpages and the vertical axis lists the type of filter fired. The size of the circle corresponds to  the proportion of elements in that domain that are of a particular element type as indicated in the corresponding vertical axis. Looking at  particular domain from top to bottom shows the composition of various elements in the web pages belonging to that domain. In most of the cases  SCRIPT,IMAGE and SUBDOCUMENT can be seen as the dominating elements. These are the elements which the Adblock plus browser addon counts as an advertisement as a part of its statistics gathering.
\begin{figure}[h]
	\centering
	\epsfig{file=figures/abp_filters.png, width=0.50\textwidth}
	\vspace*{-0.5cm}
	\caption{\textbf{Filter Types in Adblock Plus}}
	\label{fig:abp-filters}
	\vspace*{-0.5cm}
\end{figure}

Figure~\ref{fig:growth}, indicates the growth of whitelist since 2012. The aacceptable ads program had initially 132 filters in it. Over the years the whitelist has grown enormously in  size.  Over 3600 filters were added  between 2012 to 2013 and the list has been growing. As of March 2016, the total number of  filters in the whitelist is 6570.
\begin{figure}[p]
	\centering
	\epsfig{file=figures/growth.png, width=0.50\textwidth}
	\vspace*{-0.5cm}
	\caption{\textbf{Growth of Whitelist.}}
	\label{fig:growth}
	\vspace*{-0.5cm}
\end{figure}

Unsurprisingly, one can see an increasing trend in the number of ads allowed by adblock plus.
Figure~\ref{fig:ads-allowed}, shows the trend in the ads allowed over the years.
\begin{figure}[p]
	\centering
	\epsfig{file=figures/alexa_ads.png, width=0.50\textwidth}
	\vspace*{-0.5cm}
	\caption{\textbf{Number of Ads allowed in the Alexa top 200.}}
	\label{fig:ads-allowed}
	\vspace*{-0.5cm}
\end{figure}

Fig~\ref{fig:block-allow} shows the difference between the number of  elements  blocked  versus the number of elements allowed for three different  combuinations of the filters.
It can be seen that the list with only the element hiding filter does poorly in blocking the  elements as it does not have the capabilities of blocking URLS.
As expected the easylist+whitelist combination allows more ads there by allowing more elements.
\begin{figure}
	\centering
	\epsfig{file=figures/exp_comparison.png, width=0.50\textwidth}
	\vspace*{-0.5cm}
	\caption{\textbf{Blocked vs Allowed for filter combinations}}
	\label{fig:block-allow}
	\vspace*{-0.5cm}
\end{figure}
\section{Conclusion and Future work}\label{sec:conclusions}

We did not set up connections with an actual ad server - in the future, that would provide more opportunities for fuzzing the page so that AdBlock Plus does not recognize them as ads.

Few of our observations:
1. The Acceptable ads criteria mention that the image should not exceed certain dimensions. However, it is not strongly enforced. Most of the filters that block images based on size are URL blocking filters. Example: http://www.prepperwebsite.com/wp-content/uploads/2014/01/JMBullion_0107_728x90.gif is blocked because the '_728x90' pattern is matched in the blacklist. It would be trivial to change the URL to reflect a different size, or no size at all, so that AdBlock Plus does not recognize it as an ad. Other examples from easylist.txt

_728_x_90_
_728by90_
_728x-90.
_728x150.
_728x60.
_728x90&
_728x90-
_728x90.

This indicates that size is not a major factor most of the cases while blocking the elements.

2. Our bit.ly shortening service worked for element hiding filters but not matching filters because of the actual redirection. Our proxy server solution could be fleshed out in a comprehensive manner (by targeted caching, for example) by the actual websites. Some other similar approaches:
1. Have the server have a list of dynamic CNAME mappings so that it becomes difficult to track them all in the blacklist
2. Since it matches only the URL, the server could provide the IP address instead of the DNS name. However, this removes any geo-location based ad approaches, unless the server already has location information of the client.

AdBlock Plus relies on manual feedback whether the criteria are satisfied before adding the page to the non-intrusive ads list.
Is both restrictive and incorrect. From all this, while the criteria for acceptable ads are nominal unless there is a well-established way to enforce that ads follow the criteria and automatically block ads that don't, the whitelisting and blacklisting process would continue a manual process of companies paying money to whitelist the sites, assuming the ads they serve follow the criteria for acceptable ads.



It would also be interesting to see how other adblock plugins compare to AdBlock Plus in terms of blocking ads. Our initial approach was to investigate the working of AdBlock for Chrome as well. It is considered the second most popular ad block plugin after AdBlock Plus. However, it was sold and it has joined the AdBlock Plus's Acceptable ads program.
 www.ghacks.net/2015/10/02/adblock-for-chrome-sold-joins-adblock-plus-acceptable-ads-program/, so we speculate it would have given similar results.
 
 A new blocker, uBlock Origin was branched out of AdBlock Plus - not using Acceptable Ads and claims to consume lesser CPU and network traffic compared to AdBlock Plus. It would be interesting to validate those claims.

The future work is interesting. Eyeo, the company behind AdBlock Plus in wake of criticism over its unwillingness to disclose the companies who have paid money, has established an independent board to oversee acceptable ads.

Other avenues for compensating advertising losses to publishers are also arising. Brave, a new browser that claims to remove most intrusive ads has brought up the notion of micropayments to be used for compensation to the publishers in place of ads, via a Bitcoin based browser wallet. Recently, following the trend, AdBlock Plus teamed up with Flattr to help voluntary readers pay publishers by tracking the users' browsing activity and distributing the money based on the engagement of users with different websites.

This also begs a larger question: Can there be any automatic way of determining if a blocked ad is non-intrusive or not?
There are three possible ways: 
Visual inspection of ads in page to see if there are simiarites, or if they violate some aesthetic criteria.
Human labeling by crowdsourcing possibly via Amazon's Mechanical Turk and see if we can obtain users\' subjective analysis of what constitutes an acceptable ad, and looking at the DOM structure to identify common patterns between ads that are acceptable.



%\end{document}  % This is where a 'short' article might terminate
% The following two commands are all you need in the
% initial runs of your .tex file to
% produce the bibliography for the citations in your paper.
\bibliographystyle{abbrv}
\bibliography{sigproc}  % sigproc.bib is the name of the Bibliography in this case
% You must have a proper ".bib" file
%  and remember to run:
% latex bibtex latex latex
% to resolve all references
%
% ACM needs 'a single self-contained file'!
%

\end{document}
