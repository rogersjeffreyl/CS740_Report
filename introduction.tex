\section{Introduction}
Content and services which are offered for free on the Internet are sometimes monetized through online advertisements.
This relies on the implicit understanding that the viewers who consume free content do so at the cost of viewing ads.
However, recent times have seen the  rise of ad blockers that control the display of  advertisements on websites. These adblockers rely on  a list of filters for blocking or allowing advertisements.
In 2011 adblock plus started monetization of the whitelist by launching what is known was an "Acceptable ads" program.As a part of the acceptable ads program adblock plus proposed  that ads  displayed on a page have to adhere to. A publisher adhering to this crtieria can pay a fee to adblock plus and get themselves whitelisted. It is known that the e-commerce and advertising giants like Amazon, Google and Microsoft paid an undisclosed sum to adblock plus to get themselves whitelisted.

While the number of users of adblock plus continues to grow, on the other hand the number of exception filters added to the whitelist has also grown over the years.
This raises the following questions.
\begin{itemize}
\item [Q1.] How does whitelisting impact the number of ads blocked? \label{q:q1}
\item [Q2.] Does Adblock Plus allow ads even if they do not satisfy the criteria of an acceptable ad? \label{q:q2}
\item [Q3.] Does adblock Plus block ads even if they satisfy the four criteria? \label{q:q3}
\item [Q4.] As more sites are added to the whitelist when can adblock plus be deemed in-effective? \label{q:q4}
\end{itemize}
We attempt to answer Q1,Q2 and Q3 in our project by performing crawling web pages from different sources and analyzing the behavior of adblock plus on these webpages.

